% This is "sig-alternate.tex" V2.1 April 2013
% This file should be compiled with V2.5 of "sig-alternate.cls" May 2012
%
% This example file demonstrates the use of the 'sig-alternate.cls'
% V2.5 LaTeX2e document class file. It is for those submitting
% articles to ACM Conference Proceedings WHO DO NOT WISH TO
% STRICTLY ADHERE TO THE SIGS (PUBS-BOARD-ENDORSED) STYLE.
% The 'sig-alternate.cls' file will produce a similar-looking,
% albeit, 'tighter' paper resulting in, invariably, fewer pages.
%
% ----------------------------------------------------------------------------------------------------------------
% This .tex file (and associated .cls V2.5) produces:
%       1) The Permission Statement
%       2) The Conference (location) Info information
%       3) The Copyright Line with ACM data
%       4) NO page numbers
%
% as against the acm_proc_article-sp.cls file which
% DOES NOT produce 1) thru' 3) above.
%
% Using 'sig-alternate.cls' you have control, however, from within
% the source .tex file, over both the CopyrightYear
% (defaulted to 200X) and the ACM Copyright Data
% (defaulted to X-XXXXX-XX-X/XX/XX).
% e.g.
% \CopyrightYear{2007} will cause 2007 to appear in the copyright line.
% \crdata{0-12345-67-8/90/12} will cause 0-12345-67-8/90/12 to appear in the copyright line.
%
% ---------------------------------------------------------------------------------------------------------------
% This .tex source is an example which *does* use
% the .bib file (from which the .bbl file % is produced).
% REMEMBER HOWEVER: After having produced the .bbl file,
% and prior to final submission, you *NEED* to 'insert'
% your .bbl file into your source .tex file so as to provide
% ONE 'self-contained' source file.
%
% ================= IF YOU HAVE QUESTIONS =======================
% Questions regarding the SIGS styles, SIGS policies and
% procedures, Conferences etc. should be sent to
% Adrienne Griscti (griscti@acm.org)
%
% Technical questions _only_ to
% Gerald Murray (murray@hq.acm.org)
% ===============================================================
%
% For tracking purposes - this is V2.0 - May 2012

\documentclass{sig-alternate-05-2015}
\usepackage{xcolor}
\usepackage{xspace}
\newcommand\todo[1]{\textcolor{red}{** #1}}
\newcommand{\pooda}{\textsc{POOD}\xspace}

\newcommand{\pood}{\textsc{Porting OpenACC To OpenMP Directives}\xspace}

\newcommand{\squishlist}{
    \begin{list}{$\bullet$}
     { \setlength{\itemsep}{0pt}      \setlength{\parsep}{3pt}
       \setlength{\topsep}{3pt}       \setlength{\partopsep}{0pt}
       \setlength{\leftmargin}{1.5em} \setlength{\labelwidth}{1em}
       \setlength{\labelsep}{0.5em} } }

\begin{document}


% Copyright
\setcopyright{acmcopyright}
%\setcopyright{acmlicensed}
%\setcopyright{rightsretained}
%\setcopyright{usgov}
%\setcopyright{usgovmixed}
%\setcopyright{cagov}
%\setcopyright{cagovmixed}


% DOI
\doi{10.475/123_4}

% ISBN
\isbn{123-4567-24-567/08/06}

%Conference
\conferenceinfo{PLDI '13}{June 16--19, 2013, Seattle, WA, USA}

\acmPrice{\$15.00}

%
% --- Author Metadata here ---
\conferenceinfo{WOODSTOCK}{'97 El Paso, Texas USA}
%\CopyrightYear{2007} % Allows default copyright year (20XX) to be over-ridden - IF NEED BE.
%\crdata{0-12345-67-8/90/01}  % Allows default copyright data (0-89791-88-6/97/05) to be over-ridden - IF NEED BE.
% --- End of Author Metadata ---

\title{Toward Automatic Translation from OpenACC to OpenMP 4}
%\subtitle{[Extended Abstract]
%\titlenote{A full version of this paper is available as
%\textit{Author's Guide to Preparing ACM SIG Proceedings Using
%\LaTeX$2_\epsilon$\ and BibTeX} at
%\texttt{www.acm.org/eaddress.htm}}}
%
% You need the command \numberofauthors to handle the 'placement
% and alignment' of the authors beneath the title.
%
% For aesthetic reasons, we recommend 'three authors at a time'
% i.e. three 'name/affiliation blocks' be placed beneath the title.
%
% NOTE: You are NOT restricted in how many 'rows' of
% "name/affiliations" may appear. We just ask that you restrict
% the number of 'columns' to three.
%
% Because of the available 'opening page real-estate'
% we ask you to refrain from putting more than six authors
% (two rows with three columns) beneath the article title.
% More than six makes the first-page appear very cluttered indeed.
%
% Use the \alignauthor commands to handle the names
% and affiliations for an 'aesthetic maximum' of six authors.
% Add names, affiliations, addresses for
% the seventh etc. author(s) as the argument for the
% \additionalauthors command.
% These 'additional authors' will be output/set for you
% without further effort on your part as the last section in
% the body of your article BEFORE References or any Appendices.

\numberofauthors{8} %  in this sample file, there are a *total*
% of EIGHT authors. SIX appear on the 'first-page' (for formatting
% reasons) and the remaining two appear in the \additionalauthors section.
%
\author{
% You can go ahead and credit any number of authors here,
% e.g. one 'row of three' or two rows (consisting of one row of three
% and a second row of one, two or three).
%
% The command \alignauthor (no curly braces needed) should
% precede each author name, affiliation/snail-mail address and
% e-mail address. Additionally, tag each line of
% affiliation/address with \affaddr, and tag the
% e-mail address with \email.
%
% 1st. author
\alignauthor
Ben Trovato\titlenote{Dr.~Trovato insisted his name be first.}\\
       \affaddr{Institute for Clarity in Documentation}\\
       \affaddr{1932 Wallamaloo Lane}\\
       \affaddr{Wallamaloo, New Zealand}\\
       \email{trovato@corporation.com}
% 2nd. author
\alignauthor
G.K.M. Tobin\titlenote{The secretary disavows
any knowledge of this author's actions.}\\
       \affaddr{Institute for Clarity in Documentation}\\
       \affaddr{P.O. Box 1212}\\
       \affaddr{Dublin, Ohio 43017-6221}\\
       \email{webmaster@marysville-ohio.com}
% 3rd. author
\alignauthor Lars Th{\o}rv{\"a}ld\titlenote{This author is the
one who did all the really hard work.}\\
       \affaddr{The Th{\o}rv{\"a}ld Group}\\
       \affaddr{1 Th{\o}rv{\"a}ld Circle}\\
       \affaddr{Hekla, Iceland}\\
       \email{larst@affiliation.org}
\and  % use '\and' if you need 'another row' of author names
% 4th. author
\alignauthor Lawrence P. Leipuner\\
       \affaddr{Brookhaven Laboratories}\\
       \affaddr{Brookhaven National Lab}\\
       \affaddr{P.O. Box 5000}\\
       \email{lleipuner@researchlabs.org}
% 5th. author
\alignauthor Sean Fogarty\\
       \affaddr{NASA Ames Research Center}\\
       \affaddr{Moffett Field}\\
       \affaddr{California 94035}\\
       \email{fogartys@amesres.org}
% 6th. author
\alignauthor Charles Palmer\\
       \affaddr{Palmer Research Laboratories}\\
       \affaddr{8600 Datapoint Drive}\\
       \affaddr{San Antonio, Texas 78229}\\
       \email{cpalmer@prl.com}
}
% There's nothing stopping you putting the seventh, eighth, etc.
% author on the opening page (as the 'third row') but we ask,
% for aesthetic reasons that you place these 'additional authors'
% in the \additional authors block, viz.
\additionalauthors{Additional authors: John Smith (The Th{\o}rv{\"a}ld Group,
email: {\texttt{jsmith@affiliation.org}}) and Julius P.~Kumquat
(The Kumquat Consortium, email: {\texttt{jpkumquat@consortium.net}}).}
\date{30 July 1999}
% Just remember to make sure that the TOTAL number of authors
% is the number that will appear on the first page PLUS the
% number that will appear in the \additionalauthors section.

\maketitle
\begin{abstract}

\end{abstract}


%
% The code below should be generated by the tool at
% http://dl.acm.org/ccs.cfm
% Please copy and paste the code instead of the example below. 
%
\begin{CCSXML}
<ccs2012>
 <concept>
  <concept_id>10010520.10010553.10010562</concept_id>
  <concept_desc>Computer systems organization~Embedded systems</concept_desc>
  <concept_significance>500</concept_significance>
 </concept>
 <concept>
  <concept_id>10010520.10010575.10010755</concept_id>
  <concept_desc>Computer systems organization~Redundancy</concept_desc>
  <concept_significance>300</concept_significance>
 </concept>
 <concept>
  <concept_id>10010520.10010553.10010554</concept_id>
  <concept_desc>Computer systems organization~Robotics</concept_desc>
  <concept_significance>100</concept_significance>
 </concept>
 <concept>
  <concept_id>10003033.10003083.10003095</concept_id>
  <concept_desc>Networks~Network reliability</concept_desc>
  <concept_significance>100</concept_significance>
 </concept>
</ccs2012>  
\end{CCSXML}

\ccsdesc[500]{Computer systems organization~Embedded systems}
\ccsdesc[300]{Computer systems organization~Redundancy}
\ccsdesc{Computer systems organization~Robotics}
\ccsdesc[100]{Networks~Network reliability}


%
% End generated code
%

%
%  Use this command to print the description
%
\printccsdesc

% We no longer use \terms command
%\terms{Theory}

\keywords{ACM proceedings; \LaTeX; text tagging}

\section{Introduction}
\todo{1 page - introduction/summary of the work - Dr. Overbey}

\section{Motivation}
\subsection{History of OpenACC and OpenMP}
\todo{Alexander}
\subsection{Future of OpenACC and OpenMP}
\todo{Alexander}
\subsection{Relevance to XSEDE}
\todo{Dr. Overbey}

\section{Overview}
\subsection{OpenACC Directives}
\todo{Alexander}
\subsection{OpenMP 4 Directives}
\todo{Alexander}

\section{Translation}
\todo{1 page including screenshot(s) - description of our tool: what directives it handles, how it works, where to get it - Nawrin}
The major compilers supporting OpenACC are Portland Group (PGI), Cray, and CAPS. On the other hand, Intel has decided not to support OpenACC until accelerators are fully integrated into OpenMP. We describe an automated approach to assist programmers in converting OpenACC to OpenMP, so that same syntax could be used for both Cray and Intel compiler while maintaining the performance. We describe a refactoring (POOD) for this purpose.

\subsection{Translation Tool}
We prototyped the refactoring in a PLDT OpenACC Eclipse plug-in (C analyses and refactorings for OpenACC). Our tool support various data flow analyses, dependence analysis, refactorings, and parser for OpenACC directives. It provides an user interface (UI) for refactorings. Other than UI, it also supports command line interface (CLI), so one can run the refactorings outside Eclipse. 

Our tool does not handle all the OpenACC directives. We transformed the directives present in EPCC level 1 OpenACC benchmark except ``kernel" directive. When user invoke a refactoring on a C source file, refactoring engine performs the precondition checking. If preconditions are violated, it shows appropriate error message to user. Then analyses and source translations are executed.
  
\subsection{Porting OpenACC To OpenMP Directives (POOD)}
The \pood refactoring converts OpenACC directives to OpenMP 4.0 directives.

\textbf{Motivation:} While porting OpenACC to OpenMP it is necessary for directives to be compatible. Without automated tooling, a programmer has to manually transform each directives.

\textbf{Precondition:} A programmer selects a C source file and invokes the \pooda transformation. The following precondition is checked:

\begin{itemize}
\item Source file contains OpenACC directives.
\item OpenACC directives in source code supported by our transformation tool.
\end{itemize}  

\textbf{Mechanism:} The refactoring introduces OpenMP directives that correspond to OpenACC directives.

POOD first determines all the preprocessor include statements and check whether there is any ``openacc.h". If openacc header presents POOD removes it.\todo{need to remove openacc functions}. From the precondition checking we have the OpenACC preprocessor pragma statements that we want to transform. For each directives Abstract Syntax Tree gives the exact position. POOD removes the existing pragma statment and inserts OpenMP directives on that position based on the directive clause. Table \ref{table:1} lists the OpenACC and corresponding OpenMP clauses that POOD transforms. For any OpenACC data directives with copy, copyin, or copyout clause POOD performs NameBinding analyses\todo{fix} for the variables to find out their declaration. If these variables are declared globally then POOD changes them to local. When all the transformations have completed, POOD writes the changes to file.


\begin{table}[t]
	\begin{minipage}{0.47\textwidth}
%\small
\caption{Equivqlent OpenACC and OpenMP clause}
\centering
\begin{tabular}{ | l@{\hspace{0.1cm}}|  @{\hspace{0.4cm}}r@{\hspace{0.1cm}}|} 
 \hline
 \hspace*{0.4cm} OpenACC Clause & OpenMP Clause\hspace*{0.6cm}\\
 \hspace*{0.5cm} (\#pragma acc)%\footnote{The ID is used to refer to each application in the rest of the paper.} 
  & (\#pragma omp)\hspace*{0.6cm} \\
 \hline
 data & target data \\ 
 copy(A[0:n*n]) & map(tofrom:A[0:n*n]) \\
 copyin(A[0:n*n]) & map(to:A[0:n*n]) \\
 copyout(A[0:n*n]) & map(from:A[0:n*n]) \\
 create(A[0:n*n]) & map(to:A[0:n*n]) \\ 
 parallel & target/teams \\ 
 parallel loop private(t1,t2) & parallel for private(t1,t2) \\
 loop & for \\
 loop private(tmp) & for private(tmp) \\
 loop reduction(+:t1,t2) & for reduction(+:t1,t2) \\
 loop independent & for independent \\
 parallel loop & target/teams/distribute \\
 \hline
\end{tabular}

\label{table:1}
\end{minipage}
\end{table}
%\normalsize


\textbf{Example:}

\todo{0.5 page - description of the conversion algorithm (what clauses translate to what?) - Nawrin/Dr.Overbey} 
\section{Evaluation}
\todo{1 page including tables - evaluation: what code did we test it on; how many lines changed; what performance numbers did we get - Nawrin}
\subsection{Test Corpus}
\subsection{Performance}


\section{Limitations and Future Work}
\todo{1 page - limitations, including directives not handled, and future work - Dr. Overbey}

\section{Related Work}
\section{Conclusion}
\todo{Dr. Overbey}

%ACKNOWLEDGMENTS are optional
\section{Acknowledgments}


%
% The following two commands are all you need in the
% initial runs of your .tex file to
% produce the bibliography for the citations in your paper.
\bibliographystyle{abbrv}
\bibliography{sigproc}  % sigproc.bib is the name of the Bibliography in this case
% You must have a proper ".bib" file
%  and remember to run:
% latex bibtex latex latex
% to resolve all references
%
% ACM needs 'a single self-contained file'!
%
%APPENDICES are optional
%\balancecolumns
%\appendix
\end{document}
